\makeatletter
\def\input@path{{etc/}{body/}}
\makeatother

\documentclass[ALICE,manyauthors]{StrinJet}
\usepackage{StrinJet}

\begin{document}

\title{Multiplicity dependence of strange and multi-strange particle in jets in \pp collisions at \seven}
\author{authors}
\begin{abstract}
	\label{sec:Abs}
%1807.11321
Comprehensive results on the production of unidentified charged particles, $\pi^{\pm}$, $\mathrm{K^{\pm}}$, p, \kzero, \kstar, $\phi$, $\Lambda$, $\Xis$, $\Oms$ hadrons in jets in proton-proton (\pp) collisions at \seven  are presented
%1507.02091v2
with two developed color reconnection models, the new color reconnection model and the rope hadronization model, in PYTHIA 8 generator. The observables are ratios of identified hadron yields as a function of the transverse momentum ($\pT$) and the final-state activity (the charged multiplicity).


\end{abstract}


\setcounter{page}{1}


\clearpage
\section{Introduction}
\label{sec:intro}
%1601.03658v2
In heavy-ion collisions at ultra-relativistic energies, it is well established that a strongly coupled Quark-Gluon-Plasma~(QGP) is formed~\cite{Adams:2005dq, Adcox:2004mh, Arsene:2004fa, Back:2004je, Schukraft:2011na}.
Recent measurements in high multiplicity \pp, p--A and d--A collisions at different energies have revealed strong flow-like effects even in these small collision systems~\cite{Abelev:2012sk, Chatrchyan:2013eya, Khachatryan:2010gv, CMS:2012qk, Abelev:2012ola, Aad:2012gla, Aad:2013fja, Chatrchyan:2013nka, Adare:2013esx, Adams:2006nd}.
%1807.11321
In a recent letter~\cite{ALICE:2016fzo}, the ALICE Collaboration reported the multiplicity dependent enhancement of strange (\kzero, \lmb and \almb) and multi-strange (\X, \Ix, \Om and \Mo) particle in \pp collisions at \seven. As well as, those results were complemented by the measurement of $\pi^{\pm}$, $\mathrm{K}^{\pm}$, p, \pbar, \kstar and $\phi$ with ALICE~\cite{ALICE:2018pal}.


%ALICE High light describe of V0 in jet paper
In a recent study, the ALICE Collaboration has studied baryon-to-meson ratios with a new method: by studying the ratios in two parts of the events separately -- inside jets and in the event portion perpendicular to a jet cone~\cite{ALICE:2021cvd}. 

%1408.2672
In contrast to the inclusive distribution, the $\lmb/\kzero$ ratio within jets in \pp and \pPb collisions does not exhibit baryon enhancement.
It is plausible that the baryon enhancement may therefore be attributable to the soft (low $Q^{2}$) component of the collision as discussed in~\cite{Cuautle:2014yda}.
This results disfavors the hard-soft recombination models, while it is consistent with a picture in which the value of baryon/meson ratio has two independent mechanisms: i) the expansion of the soft particles of the underlying event within  a common velocity field (radial flow), and ii) the production of particles via hard parton-parton scatterings and the subsequent jet fragmentation.
%1906.03145
This comprehensive set of data does allow for a detailed test of production models.

%1606.07424v2
Such behaviour cannot be reproduced by any of the MC models commonly used, suggesting that further developments are needed to obtain a complete microscopic understanding of strangeness production and indicating the presence of a phenomenon novel in high-multiplicity pp collisions.
%1601.03658v2
%The origin of these phenomena is debated in~\cite{Shuryak:2013ke, Werner:2013ipa, Bozek:2013ska, Dumitru:2010iy, Schenke:2015aqa, Ma:2014pva, Ortiz:2013yxa}.
%It was also discussed that in small systems, mechanisms like color-reconnection may produce radial flow-like effects.

%1512.07227
%The role of strange hadron yields in searching for QGP was pointed out at an early stage~\cite{Rafelski:1982pu}.
%It was subsequently found that in high energy nucleus-nucleus (A--A) collisions at the Super Proton Synchrotron (SPS), the Relativistic Heavy Ion Collider (RHIC) and the Large Hadron Collider (LHC) the abundances of strange and multi-strange baryons are compatible with those from thermal statistical model calculations~\cite{Andersen:1998vu, NA49:2002fxd, NA57:2004nxc, NA49:2003bok, STAR:2003jis, STAR:2006egk, STAR:2007cqw, ALICE:2013xmt }

%1204.0282v3
%The multi-strange baryons, $\Omega$~(sss) and $\Xi$~(dss), are particularly important in high energy particle and nuclear physics due to their dominant strange quark~(s-quark) content. The initial state colliding projectiles contain no strange valence quark, therefore all particles with non-zero strangeness quantum number are created in the course of the collision.


%1901.07447
The theoretical picture of collective effects in heavy ion collisions is vastly different from the picture known from \pp collisions. Due to the very different geometry of the two system types, interactions in the final state of the collision become dominant in heavy ion collisions, while nearly absent in \pp collisions.

%1612.05132
To provide a description of the hadrochemistry in the underlying event of \pp collisions, it has been suggested a "rope hadronization" model~\cite{Bierlich:2014xba}, based on work by Biro, Knoll and Nielsen~\cite{Biro:1984cf}. This model provides corrections to the string hadronization model, by allowing strings overlapping in transverse space to act coherently as a "rope". The model is implemented in the DIPSY event generator~\cite{Flensburg:2011kk}, which provides a dynamical picture of the event structure in impact parameter space, allowing for a calculation of the colour field strength in each small rope segment\footnote{A Lund string is in its simplest form, a straight piece stretched between a quark and an anti-quark, or a colour triplet and anti-triplet. As gluons are added to the string, they act as point-lik "kinks" on the string, carrying energy and momentume~\cite{Andersson:1979ij}. We will denote all straight pieces between gluons or (anti)quarks string segments. A $q - g - \overline{q}$ string thus has two segments.}. This formalism also includes all fluctuations. The colour field is characterized by two quantum numbers ${p, q}$, which together signifies its SU(3) multiplet structure. Lattice calculations have shown~\cite{Bali:2000un}, that the string tension -- energy per unit length -- scales with the quadratic Casimir operator of the multiplet, such that the ratio of the enhanced rope tension ($\widetilde{\kappa}$) to the triplet string tension in vacuum $\kappa$ is:


%1507.02091v2
In this study we consider two of the models: the new colour reconnection (CR) model~\cite{Christiansen:2015yqa} and the colour rope model~\cite{Biro:1984cf} in the PYTHIA 8 generator.
\section{Models}
\label{sec:model}
%describe the models we used (Rope and CR)

\section{Compare to data}
\label{sec:com2da}


\section{Predictions}
\label{sec:predic}

\section{Summary}
\label{sec:sum}


\newenvironment{acknowledgement}{\relax}{\relax}
%\begin{acknowledgement}
%\section*{Acknowledgements}
%\input{acknowledgements.tex}

%\end{acknowledgement}

\bibliographystyle{etc/utphys}
\bibliography{StrinJet}

\newpage
\appendix
\section{Model parameters}
\label{app:modpara}

\begin{table}[ht]
\label{tab:CRparameter}
  \begin{center}
  \begin{tabular}{|c|c|}
	\hline
	  Parameters & Values \\
	\hline 
	MultiPartonInteractions:pT0Ref &  2.15\\ 
	BeamRemnants:remnantMode & 1 \\
	BeamRemnants:saturation & 5 \\
	ColourReconnection:reconnect & on \\
	ColourReconnection:mode & 1 \\
	ColourReconnection:allowDoubleJunRem & off \\
	ColourReconnection:m0 & 0.3  \\
	ColourReconnection:allowJunctions & on \\
	ColourReconnection:junctionCorrection & 1.2 \\;
	ColourReconnection:timeDilationMode & 2 \\
	ColourReconnection:timeDilationPar & 0.18\\ 
	\hline 
  \end{tabular} 
  \caption{Colour reconnection model parameters}
  \end{center}
\end{table}

\begin{table}[ht]
	\label{tab:Ropeparameter}
	\begin{center}
		\begin{tabular}{|c|c|}
			\hline
			Parameters & Values \\
			\hline 
			Ropewalk:RopeHadronization & on \\
			Ropewalk:doShoving & on  \\
			Ropewalk:tInit & 1.5 \\
			Ropewalk:deltat & 0.05 \\
			Ropewalk:tShove & 0.1 \\
			Ropewalk:gAmplitude & 0. \\
			Ropewalk:doFlavour & on \\
			Ropewalk:r0 & 0.5 \\
			Ropewalk:m0 & 0.2 \\
			Ropewalk:beta & 0.1 \\
			\hline 
		\end{tabular} 
		\caption{Rope hadronization model parameters}
	\end{center}
\end{table}
%\input{} % put your appendices here (if any)
 
\end{document}
